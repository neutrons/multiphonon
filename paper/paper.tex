\documentclass{article}
\usepackage[utf8]{inputenc}
\usepackage{authblk}
\usepackage{graphicx}


\title{Multiphonon: Phonon Density of States tools for Inelastic Neutron Scattering Powder Data}
\author[1]{Jiao Y. Y. Lin \thanks{Corresponding author. email: linjiao@ornl.gov}}
\author[1]{Fahima Islam}
\author[2]{Max Kresch}
\affil[1]{Neutron Scattering Division, Oak Ridge National Lab}
\affil[2]{mkresch@gmail.com}

\date{September 2017}

\begin{document}

\maketitle

\section{Summary}\label{summary}

The multiphonon python package calculates phonon density of states,
a reduced representation of vibrational property of condensed matter (see, for example,
Section "Density of Normal Modes" in
Chapter 23 "Quantum Theory of the Harmonic Crystal" of \cite{ashcroftmermin}),
from
inelastic neutron scattering (see, for example \cite{FultzINSbook})
spectrum from a powder sample.
Inelastic neutron spectroscopy (INS) is a probe of excitations in solids of
vibrational or magnetic origins.
In INS, neutrons can lose(gain) energy
to(from) the solid in the form of quantized lattice vibrations -- phonons.
Measuring phonon density of states is usually the first step
in determining the phonon properties of a material experimentally.
Phonons play a very important role in understanding the physical properties of a solid,
including thermal conductivity and electrical conductivity.
Hence, INS is an important tool for studying thermoelectric materials \cite{budai2014, lichen2015},
where
low thermal conductivity and high electirical conductivity are desired.
Study of phonon entropy also made important contributions to
the research of thermal dynamics and phase stability of materials
\cite{FULTZ2010, bogdanoff2002phonon, swan2006vibrational}.

The algorithm implemented in this package is a self-consistent,
iterative procedure that finishes when
the measured INS spectrum can be accounted for by
the one-phonon scattering, multi-phonon scattering, and multiple
scattering from the deduced phonon density of states, under the
incoherent approximation (Appendix of \cite{KreschNickel2007} and
Section 6.5 ``Calculation of Multiphonon Scattering'' of
\cite{FultzINSbook}).

\begin{figure}
  \includegraphics[scale=0.25]{sqe2dos}
  \caption{The multiphonon package takes the inelastic neutron scattering spectrum, shown on the left, and produces the phonon density of states shown on the right.}
\end{figure}


\section{Notice of Copyright}\label{notice-of-copyright}

This manuscript has been authored by UT-Battelle, LLC under Contract No.
DE-AC05-00OR22725 with the U.S. Department of Energy. The United States
Government retains and the publisher, by accepting the article for
publication, acknowledges that the United States Government retains a
non-exclusive, paid-up, irrevocable, worldwide license to publish or
reproduce the published form of this manuscript, or allow others to do
so, for United States Government purposes. The Department of Energy will
provide public access to these results of federally sponsored research
in accordance with the DOE Public Access Plan
(http://energy.gov/downloads/doe-public-access-plan).

\section{Acknowledgements}\label{acknowledgements}
This work is sponsored by the Laboratory Directed Research and
Development Program of Oak Ridge National Laboratory, managed by
UT-Battelle LLC, for DOE. Part of this research is supported by the U.S.
Department of Energy, Office of Science, Office of Basic Energy
Sciences, User Facilities under contract number DE-AC05-00OR22725.

We thank Douglas Abernathy, Jennifer Niedziela, Iyad Al-Qasir,
Dipanshu Bansal, and Chen Li for stimulating discussions.


\bibliographystyle{unsrt}
\bibliography{paper}

\end{document}
